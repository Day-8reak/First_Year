\documentclass{article}
\usepackage{graphicx} 
\usepackage{minted}
% Required for inserting images

\title{Report on Sudoku Solver}
\author{Maxwell Huang}
\date{March 2024}

\begin{document}

\maketitle

\section{Entire code file}

\begin{minted}{c}
#include <stdio.h>
#include <stdbool.h> // import for bool
#include <math.h> // import for math library

// Code : Here include your necessary library(s)
// Code : Write your global variables here, like:
#define N 9

int count = 0;

/* Code: Write your functions here, or the declaration of the head/
For example write the recursive function solveSudoku(), like:
*/


void printGrid(int arr[N][N])
{
    int i;
    
	for(i = 0; i < N; i++){
        for(int j = 0; j < N; j++)
        {
            printf("%d ", arr[i][j]);
            
            if (j % 3 == 2 && j != 0)
            {
                printf("| ");
            }
        }
        printf("\n----------------------- \n");
	}
}

bool isValid(int x, int y, int n, int grid[N][N])
{
    if (n > 9 || n < 1)
    {
        return false;
    }

    //checking column
    for (int i = 0; i < 9; i++)
    {
        if (n == grid[i][y])
        {
            return false;
        }
    }
    
    //checking row
    for (int i = 0; i < 9; i++)
    {
        if (n == grid[x][i])
        {
            return false;
        }
    }


    //checking subgrid
    int x0 = floor(x/3)*3;
    int y0 = floor(y/3)*3;

    for (int i = 0; i < 3; i++)
    {
        for (int j = 0; j < 3; j++)
        {
            if (grid[x0 + i][y0 + j] == n)
            {
                return false;
            }   
        }
    }
    /* Somehow we need to find out either where we are in the subgrid and
     then measure agains the other numbers in that subgrid
     OR
     We have to find out which grid we're in 
    */
    return true;
}


bool solveSudoku(int puzzle[][9], int row, int col)
{
    int i;
    if(row<9 && col<9)
    {
        if(puzzle[row][col] != 0)
        {
            if((col+1)<9) return solveSudoku(puzzle, row, col+1);
            else if((row+1)<9) return solveSudoku(puzzle, row+1, 0);
            else return 1;
        }
        else
        {
            for(i=0; i<9; ++i)
            {
                if(isValid(row, col, i+1, puzzle))
                {
                    puzzle[row][col] = i+1;
                    if((col+1)<9)
                    {
                        if(solveSudoku(puzzle, row, col +1)) return 1;
                        else puzzle[row][col] = false;
                    }
                    else if((row+1)<9)
                    {
                        if(solveSudoku(puzzle, row+1, 0)) return 1;
                        else puzzle[row][col] = false;
                    }
                    else return true;
                }
            }
        }
        return false;
    }
    else return true;
}

/* This is my second attempt at a solvesudoku function which didn't quite work

bool solveSudoku(int gridS[N][N], int r, int c)
{
    count = count + 1;
    if (r == 9)
    {
        return true;
    }
    else if (c == 9)
    {
        solveSudoku(gridS, r+1, 0);
    }
    else if (gridS[r][c] != 0)
    {
        solveSudoku(gridS, r, c+1);
    }
    else
    {
        for (int i = 0; i < 9; i++)
        {
            if (isValid(r, c, i, gridS))
            {
                gridS[r][c] = i;
                if (solveSudoku(gridS, r, c+1))
                {
                    return true;
                }
                gridS[r][c] = 0;
            }
        }
        return false;
    } 



    /*This is my first attempt at a solvesudoku function
    for (int i = 0; i < 9; i++)
    {
        for (int j = 0; j < 9; j++)
        {
            if (gridS[i][j] == 0)
            {
                for (int l = 1; l < 10; l++)
                {
                    if (isValid(i, j, l, gridS))
                    {
                        gridS[i][j] == l;
                        solveSudoku(gridS);
                        gridS[i][j] = 0;
                    }
                }
            }   
        }   
    } 
}
*/


int main()
{
    //This is hard coding to receive the "grid"
    int grid [N][N] = 
    {
        {0, 2, 0,    0, 0, 0,    0, 0, 0},
        {0, 0, 0,    6, 0, 0,    0, 0, 3},
        {0, 7, 4,    0, 8, 0,    0, 0, 0},
        
        {0, 0, 0,    0, 0, 3,    0, 0, 2},
        {0, 8, 0,    0, 4, 0,    0, 1, 0},
        {6, 0 ,0,    5, 0, 0,    0, 0, 0},
        
        {0, 0, 0,    0, 1, 0,    7, 8, 0},
        {5, 0, 0,    0, 0, 9,    0, 0, 0},
        {0, 0, 0,    0, 0, 0,    0, 4, 0},
    };


    printf("The input Sudoku puzzle: \n");
    // "Print is a function we define to print the grid"
    printGrid(grid);

    if (solveSudoku(grid, 0, 0))
    {
        // If the puzzle is solved then:
        printf("Solution found after %d iterations: \n", count);
        printGrid(grid);
    }
    else
    {
        printf("No solution exists. \n");
    }
    return 0;
}   
\end{minted}


\section{functions and their purposes}
Let's look at each function step by step:

First we have the printGrid function:
\begin{minted}{c}
void printGrid(int arr[N][N])
{
    int i;
    
	for(i = 0; i < N; i++){
            for(int j = 0; j < N; j++)
            {
                printf("%d ", arr[i][j]);
                
                if (j % 3 == 2 && j != 0)
                {
                    printf("| ");
                }
            }
         printf("\n----------------------- \n");
	}
}  
\end{minted}
The printGrid function here is essential because it
is the function designed to print the sudoku puzzle.

How it does this is simply by looping through 2 for loops
of length 9 and then printing the number at that location in the
9 x 9 for loop. On top of that the function prints a |
every 3 numbers, and prints a "-----------------------".
These additional measures make sure that the sudoku puzzle is more legible
by seperating each line from one another, and isolating every
3 x 3 subgrid.



Second, we have the isvalid function:
\begin{minted}{c}
bool isValid(int x, int y, int n, int grid[N][N])
{
    if (n > 9 || n < 1)
    {
        return false;
    }

    //checking column
    for (int i = 0; i < 9; i++)
    {
        if (n == grid[i][y])
        {
            return false;
        }
    }
    
    //checking row
    for (int i = 0; i < 9; i++)
    {
        if (n == grid[x][i])
        {
            return false;
        }
    }

    //checking subgrid
    int x0 = floor(x/3)*3;
    int y0 = floor(y/3)*3;
    for (int i = 0; i < 3; i++)
    {
        for (int j = 0; j < 3; j++)
        {
            if (grid[x0 + i][y0 + j] == n)
            {
                return false;
            }   
        }
    }
    
    return true;
}
\end{minted}
The isValid function checks if it's possible to put a give number n
into the position (x,y) on a sudoku grid.

It does by checking the three conditions of sudoku:
First, it checks the row of position (x,y) and makes sure
that there is no other position with the same value n.
To check the row, we loop 9 times checking if (i, y) == n. 
If it does equal n then we return false, otherwise we move
to the next step.


Second, it checks the column of position (x,y) and makes sure
that there is no other position in that column with the same value n.
We do this pretty much the same way as we checked the row


Third, it checks the subgrid of the position, and makes sure
that there is no other position in that subgrid with the same 
value as n.
For this algorithm, I had to be a bit clever.
The core idea was to take the location (x,y), and then find
the square at the upper left square of the subgrid we're in.
From there, we can have a nested for loop both iterating 3 times
to get to any point in the subgrid to compare the values



\section{Algorithm implemented}
    Finally, we have the most important part of the code,
that being the actual algorithm.
The core idea behind the algorithm was to loop through every single
index in the array. Then for each of the empty indices we would try
to see if each value from 1 - 9 would work. If a value works
then we would call the function again and do the same thing for
the next index in the array. Eventually if down the line there's
no correct value that can be inputted into the next index, then 
we backtrack one iteration and try a different value for the previous index.
This entire algorithm is called backtracking, here's the code for it:

\begin{minted}{c}
bool solveSudoku(int puzzle[][9], int row, int col)
{
    int i;
    if(row<9 && col<9)
    {
        if(puzzle[row][col] != 0)
        {
            if((col+1)<9) return solveSudoku(puzzle, row, col+1);
            else if((row+1)<9) return solveSudoku(puzzle, row+1, 0);
            else return 1;
        }
        else
        {
            for(i=0; i<9; ++i)
            {
                if(isValid(row, col, i+1, puzzle))
                {
                    puzzle[row][col] = i+1;
                    if((col+1)<9)
                    {
                        if(solveSudoku(puzzle, row, col +1)) return 1;
                        else puzzle[row][col] = false;
                    }
                    else if((row+1)<9)
                    {
                        if(solveSudoku(puzzle, row+1, 0)) return 1;
                        else puzzle[row][col] = false;
                    }
                    else return true;
                }
            }
        }
        return false;
    }
    else return true;
}   
\end{minted}



\end{document}

Your report must be in a LaTeX format ( report.tex ). In your report, describe the algorithm you
have programmed, the purpose of each function, and include all your codes in Sudoku Solver.c
or Sudoku Solver GUI.c . Produce the PDF file ( report.pdf ) to make sure it is working, and
submit both of them.